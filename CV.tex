%______________________________________________________________________________________________________________________
% 

\documentclass[margin,line]{resume}
\usepackage{url}
\usepackage{hyperref}
\usepackage{fancyhdr}
\usepackage{color}
\usepackage{fancyhdr}

%______________________________________________________________________________________________________________________
\begin{document}

\name{\Large{\color{blue}{Sudipto Mukherjee}}\hspace{257pt}{\color{blue}{Curriculum Vitae}}}
\begin{resume}

    %__________________________________________________________________________________________________________________
    % Contact Information
    \section{\mysidestyle Contact\\Details}

    Building 99, 14820 NE 36th St, \hfill Email : \href{mailto:sudmukh@microsoft.com}{\nolinkurl{sudipto.ece.ju@gmail.com}} \\ 
    Redmond, WA 98052. \hfill Github: sudiptodip15,  Tel: (206)747-4808    
     %__________________________________________________________________________________________________________________
    % Contact Information
%    \section{\mysidestyle Research\\Objective}
%
%    Solving the challenges for science in the $21^{st}$  century by leveraging advances in Artificial Intelligence, Machine Learning and Large Language models.
    \section{\mysidestyle Work Experience}
\begin{itemize}
\vspace{-0.5pt}
\item \textbf{Senior Applied Scientist, Microsoft.} \\
Microsoft Customer \& Partner Solutions (MCAPS) \hfill Nov 2023 - Present\\
 Research (Project Science Engine Graduatation to MCAPS)  \hfill Dec 2021 - Oct 2023\\
Microsoft Search, Assistance and Intelligence (MSAI). \hfill Sept 2020 - Dec 2021\\
\end{itemize}
    %__________________________________________________________________________________________________________________
    
                                
    %__________________________________________________________________________________________________________________
    % Contact Information
%    \section{\mysidestyle Objective}
%
%    Looking for an opportunity to apply my strengths in Machine Learning, Optimization and Deep Learning to a high impact problem domain.
%    %__________________________________________________________________________________________________________________
    %Education
    \section{\mysidestyle Education}
\begin{itemize}
\vspace{-0.5pt}
\item \textbf{ {Doctor of Philosophy }} (Electrical and Computer Engineering)     \\ 
University of Washington, Seattle, WA.   \\ 
\emph{Cumulative GPA} : 3.92 out of 4.0 \hfill Sept 2015 - Aug 2020
\item \textbf{ {Master of Science }} (Electrical Engineering)     \\
University of Washington, Seattle, WA.   \hfill Sept 2015 - March 2018

\item \textbf{ {Bachelor of Electronics and Telecommunication Engineering }}     \\ 
%Department of Electronics and Telecommunication Engineering, \\
Jadavpur University, Kolkata, India. \\                     
%\vspace{1mm}
%\vspace{5mm}
\emph{Cumulative GPA} : 9.67 out of 10.0 \hfill Aug 2011 - June 2015\\
%\emph{Percentage} : \textbf{90.63}\\
%\emph{Overall Rank in Class} :\textbf{ 3\textsuperscript{rd}}
\end{itemize}

\section{\mysidestyle Research Internship}
\begin{itemize}
\vspace{-4.5pt}
\item \textbf{Data and Applied Scientist Intern, Microsoft} \\  Communications Intelligence Group, MSAI, \\ (\textit{Collaboration with Microsoft Research, Redmond}). \hfill June - Sept. 2019.

\item \textbf{Data and Applied Scientist Intern, Microsoft} \\ Substrate and Query Intelligence (AI + R). \hfill June - Sept. 2018.

\item \textbf{Indo-German Exchange Scholarship (Daad-Wise)} \\ Computer Vision and Pattern Recognition Group, Institute of Computer Science, \\ Westf\"alische Wilhelms-Universit\"at M\"unster, Germany.
\hfill May - August 2014.

\end{itemize}

%______________________________________________________________________________________________________________________
\section{\mysidestyle Technical\\Skills} 
\begin{itemize}
\vspace{-0.5pt}
\item \textbf{Programming Languages} : Python (Proficient), C++, Java (Prior exposure)
\item \textbf{Machine Learning Frameworks} : PyTorch, Semantic Kernel, Langchain.
\item \textbf{Data and MLOps} : Azure Machine Learning, Databricks, SQL, PromptFlow.
\end{itemize}
%\vspace{10pt} 

 %____________________________________________________________________________________________________________
% PG research Project

\section{\mysidestyle Selected Research Projects} 

\begin{itemize} 
\vspace{-0.5pt}

\item \textbf{Meeting Summarization from Teams transcripts}
\begin{itemize}  
\vspace{-0.5pt}
%\item {\textbf {Advisor}} : \textbf{Prof. Dr. Sreeram Kannan, University of Washington, Seattle, WA.}  
\item As part of a research and incubation effort inside MSAI, built a two stage extractive and abstractive summarization solution to summarize the key discussions for recorded meetings. Large Language Model (LLM) was used to create synthetic dialogs as well as summaries for distilled model fine-tuning. Feature is currently planned for release to Teams Premium customers.
\end{itemize}

\item \textbf{Smart To-Do : Automatic Generation of To-Do List from Emails}
\begin{itemize}
\item \textit{This work was featured in online publications and AI news websites such as TechZine, Beebom and The Next Web (TNW).}
%\item {\textbf {Microsoft Internship Project}}      
\item (\textit{Microsoft Internship, Summer 2019}) Created an entire machine learning pipeline - (a) Defining problem structure and annotation guidelines, (b) Data generation through Crowdsourcing, (c) Design of extractive and abstractive summarization algorithms for To-Do item generation from emails. The generated To-Do lists were remarkably similar to those written by humans. 

\end{itemize}

\item \textbf{Clustering in Generative Adversarial Networks}
\begin{itemize}  
\vspace{-0.5pt}
%\item {\textbf {Advisor}} : \textbf{Prof. Dr. Sreeram Kannan, University of Washington, Seattle, WA.}  
\item This project found traditional Gaussian and Uniform priors to be unsuitable for clustering in GANs. We designed a new architecture, ClusterGAN, that achieves low-dimension embeddings with cluster structure and preserves interpolation properties in latent space.
\end{itemize}


\item \textbf{Estimation of Conditional Mutual Information}
\begin{itemize} 
\vspace{-0.5pt}
%\item {\textbf {Advisor}} : \textbf{Prof. Dr. Sreeram Kannan, University of Washington, Seattle, WA.} 
\item (\textit{Collaboration with IBM Research}) In this project, novel estimators for mutual information and conditional mutual information were developed using generative models and classfiers. A test for conditional independence was also built from these estimators.
\end{itemize}






%\item {\textbf{Topic}} : \textsc{Robust Community Detection with applications to genomics}
%\begin{itemize} 
%\item {\textbf {Advisor}} : \textbf{Prof. Dr. Sreeram Kannan, University of Washington, Seattle, WA.}    
%\item {\textbf {Description} } : A non-convex constrained optimization framework was adapted to detect communities in friendship-foe networks. The adversarial nodes having random connections can be removed by iterative techniques similar to robust regression. This algorithm was applied again for resolving segmental duplications. 
%\end{itemize}

\newpage

\item \textbf{Project Science Engine: Search and Recommendations}
\begin{itemize}
%\item {\textbf {Microsoft Internship Project}}      
\item As part of \href{https://www.microsoft.com/en-us/research/project/project-s/overview/}{Project Science Engine}, trained and deployed models for:\\
(a) Multimodal search to retrieve reaction recipes for a new query reaction. It involved jointly embedding reactions and known reaction recipes in a vector space using multimodal transformer based architecture. (Patent application \href{https://patents.google.com/patent/WO2024005977A1/en?oq=WO2024005977A1}{WO2024005977A1}) \\
(b) Ranking papers based on predicted impact score. Trained ranking model based on text embeddings, journal and author reputation features and created automatic pipeline in Azure to continuously rank newly published papers.
\end{itemize}

%\item \textbf{Resolving Segmental Duplications in Human Genome}
%\begin{itemize}
%%\item {\textbf {Advisor}} : \textbf{Prof. Dr. Sreeram Kannan, University of Washington, Seattle, WA.}    
%\item Structural variations in the genome can shed light on causes of diseases such as Autism, Schizophrenia and provide answers to the long standing missing heritability problem. (a) The problem was formulated as an incomplete noisy low rank matrix completion over discrete entries and solved using alternating minimization coupled with a EM-type postprocessing.  (b) A scalable approach using robust community detection for signed networks was also designed that led to improved performance.  
%\end{itemize}


\item \textbf{Content-Based Email Response Prediction}
\begin{itemize}
%\item {\textbf {Microsoft Internship Project}}      
\item (\textit{Microsoft Internship, Summer 2018}) Scalable user representations based on the content of historically received emails was used for email reply prediction on publicly available Avocado Email Collection. This approach solved the acute ``cold-start'' problem and was able to accommodate the inherent temporal nature of the problem, where users and email content constantly evolve with time.
\end{itemize}



%\item \textbf{Manifold Ranking based on regularized graph constructions}
%\begin{itemize}
%%\item {\textbf {Daad-Wise Project}}  
%%\item {\textbf {Advisor}} : \textbf{Prof. Dr. Xiaoyi Jiang, University of Muenster, Germany.}    
%\item({\textit{Daad-Wise Project}}) We explored better graph construction methods that can adequately represent the manifold, with application to image retrieval. We formulated a discrete regularized optimization and proved that a greedy algorithm could achieve the global optima.
%\end{itemize}



%\item {\textbf{Topic}} : \textsc{Deep Learning for Time Series Nanopore Sequencing Data}
%\begin{itemize}  
%\item {\textbf {CSE 547 Machine Learning for Big Data Course Project}}   
%\item {\textbf {Description} } : The goal was to infer the sequence of bases from the time series of current changes. We compared performance of classical HMM decoding with recurrent neural networks. RNN/LSTM was able to capture many of the intricacies and biases in data not explicitly modeled by HMM state transitions and thereby performed better.
%\end{itemize}


\end{itemize}


%\section{\mysidestyle Relevant Graduate Courses} 
% \begin{itemize}
% \vspace{-0.5pt}
% \item Machine Learning, Natural Language Processing, Convex Optimization, Design and Analysis of Algorithms, Probabilistic Graphical Models, Information Theory, \\ Probability and Random Processes, Computational Biology, Stochastic Processes.\\
% \textbf{Teaching Assistant} : Data Science (Spring '17), Representation Learning (Spring '18), \\ Information Theory (Winter '19), Probability Models and Inference (Spring '20).
% \end{itemize}

\section{\mysidestyle Service to Community} 
 \begin{itemize}
 \vspace{-0.5pt}
 \item \textbf{Review}: Served as reviewer of top-tier journals and conferences in the field of Machine Learning and AI including Nature Communications, IEEE TIP, ISIT, IJCAI, IEEE TSMC, SIGGRAPH and JSAIT.
 \item \textbf{Teaching Assistant} : Data Science (Spring '17), Representation Learning (Spring '18), \\ Information Theory (Winter '19), Probability Models and Inference (Spring '20).
 \end{itemize}

\section{\mysidestyle Selected Publications} 
\begin{itemize}

\item  \emph{``Smart To-Do: Automatic Generation of To-Do Items from Emails"}-  
Sudipto Mukherjee, Subhabrata Mukherjee, Marcello Hasegawa, Ahmed Hassan Awadallah, Ryen White. \, \emph{58th annual meeting of the Association for Computational Linguistics (ACL), 2020.} (17.5 $\%$ Acceptance)


\item  \emph{``CCMI : Classifier based Conditional Mutual Information Estimation"}-  
Sudipto Mukherjee, Himanshu Asnani, Sreeram Kannan.  \emph{Conference on Uncertainty in Artificial Intelligence \\ (UAI-19)}. (26 $\%$ Acceptance)


\item  \emph{``ClusterGAN : Latent Space Clustering in Generative Adversarial Networks"}-  Sudipto Mukherjee, Himanshu Asnani, Eugene Lin, Sreeram Kannan. 
\emph{33rd AAAI Conference on Artificial Intelligence (AAAI-19)}. (16.2 $\%$ Acceptance, Oral Presentation)


\item  \emph{``Resolving multicopy duplications de novo using polyploid phasing "}-  
Mark J. Chaisson$^*$, Sudipto Mukherjee$^*$, Sreeram Kannan, Evan E. Eichler. 
\emph{$21^{st}$ International Conference on Research on Computational Molecular Biology, RECOMB 2017}. (21.6 $\%$ Acceptance)

%\item  \emph{``Content-Based Email Response Prediction : Exploration and Evaluation"}-  
%\textbf{Sudipto Mukherjee}, Ke Jiang.  (\textit{arXiv:1905.01991})

\item  \emph{``A deep adversarial variational autoencoder model for dimensionality reduction in single-cell RNA sequencing analysis"}- Eugene Lin, Sudipto Mukherjee, Sreeram Kannan. \\ \textit{BMC Bioinformatics}, 2020.

\item  \emph{``Robust Community Detection for resolving segmental
duplications : The last frontier to assembly "}- Sudipto Mukherjee, Sreeram Kannan.  \emph{NeurIPS Workshop on Machine Learning in Computational Biology (MLCB)}, 2017.
 
\item  \emph{``Variants of k-regular nearest neighbor graph and their construction"}- 
Klaus Broelemann, Xiaoyi Jiang, Sudipto Mukherjee, Ananda S.Chowdhury. 
\emph{Information Processing Letters, Elsevier}, October 2017.
% Vol.126, October 2017, Pages 18-23.} 
%\url{https://doi.org/10.1016/j.ipl.2017.06.001} 

\item  \emph{``A Lagrangian Approach to Modeling and Analysis of a Crowd Dynamics"}- Sudipto Mukherjee, Debdipta Goswami, Sarthak Chatterjee.
\emph{IEEE Transactions on Systems, Man and Cybernetics: Systems}, June 2015.
%Vol.45, Issue 6, June 2015, Pages 865-876.\\
%\url{http://dx.doi.org/10.1109/TSMC.2015.2389763} 

%\item \emph{``Stability and chaos analysis of a novel swarm dynamics with applications to multi-agent systems"}- Swagatam Das, Debdipta Goswami, Sarthak Chatterjee, and Sudipto Mukherjee.  \emph{Engineering Applications of Artificial Intelligence}, April 2014.


\end{itemize}

 
\section{\mysidestyle Major Academic\\Achievements\\and\\Accolades} 
 \begin{itemize}
\item Recipient of Prof. Jnansaran Chatterjee Memorial \textit{Gold Medal} in Jadavpur University Convocation for highest aggregate score in Tele-Communication subjects. 
\item Recipient of the \textit{Supriya Basu Scholarship} for being the \textit{Best Student from Faculty of Engineering - 3\textsuperscript{rd} Year} (2014).
\item Recipient of \textit{Meera Rani Mitra Memorial Award (Gold Plated Silver Medal)} for securing the highest marks in the Department in 3\textsuperscript{rd} year Engineering (2014). %University Examination in Engineering (2014)
\item Recipient of \textit{Mamraj Agarwal Rashtriya Purashkar} in August, 2011 based on Standard 12 result from His Excellency Governor of West Bengal, India.
\end{itemize}

   
%--------------------------------------------------------------------------------------------------------------------


%_____________________________________________________________________________________________________________________
%_______________________________________________________________________________________________________
%\section{\mysidestyle References}
%\begin{itemize}
%\item \textbf{Dr. Ananda Shankar Chowdhury}\\
%Associate Professor,\\
%Electronics and Telecommunication Engineering,\\
%Jadavpur University,\\
%Kolkata, India.\\
%Email :\href{mailto:aschowdhury@etce.jdvu.ac.in}{\nolinkurl{aschowdhury@etce.jdvu.ac.in}}\\\\\\
%\item \textbf {Dr. Swagatam Das}\\
%Assistant Professor,\\
%Electronics and Communication Sciences Unit,\\
%Indian Statistical Institute,\\
%Kolkata, India.\\
%Email : \href{mailto: swagatam.das@isical.ac.in}{\nolinkurl{swagatam.das@isical.ac.in}}\\\\\\
%\item \textbf{Dr. Xiaoyi Jiang}\\
%Professor,\\
%Department of Computer Science,\\
%University of M\"unster,\\
%M\"unster, Germany.\\
%Email :\href{mailto:xjiang@uni-muenster.de}{\nolinkurl{xjiang@uni-muenster.de}}\\\\\\
%
%%\item \textbf{Swagatam Das}\\\\
%%Visiting Assistant Professor\\\\
%%Electronics and Communication Sciences Unit (ECSU),\\\\
%%Indian Statistical Institute (ISI), Kolkata - 700 108\\\\
%%Email : \href{mailto:swagatam.das@isical.ac.in}{\nolinkurl{swagatam.das@isical.ac.in}}\\
%\end{itemize}



%______________________________________________________________________________________________________________________
\end{resume}
\end{document}


%______________________________________________________________________________________________________________________
% EOF

